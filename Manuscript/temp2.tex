\documentclass[10pt,a4paper]{article}
\usepackage[top=0.85in,footskip=0.75in]{geometry}

%
\usepackage{amsmath,amssymb}

%
\usepackage{changepage}

%
\usepackage[utf8x]{inputenc}

%
\usepackage{textcomp,marvosym}

%
%\usepackage{cite}

%
\usepackage{nameref,hyperref}

%
\usepackage[right]{lineno}
\linenumbers %
\usepackage{gensymb}

%
\usepackage{microtype}
\DisableLigatures[f]{encoding = *, family = * }

%
%\usepackage[table]{xcolor}

\usepackage{mathtools}

%
\usepackage{array}

%
\usepackage{float}

%
\newcolumntype{+}{!{\vrule width 2pt}}

%
\newlength\savedwidth
\newcommand\thickcline[1]{%
	\noalign{\global\savedwidth\arrayrulewidth\global\arrayrulewidth 2pt}%
	\cline{#1}%
	\noalign{\vskip\arrayrulewidth}%
	\noalign{\global\arrayrulewidth\savedwidth}%
}

%
\newcommand\thickhline{\noalign{\global\savedwidth\arrayrulewidth\global\arrayrulewidth 2pt}%
	\hline
	\noalign{\global\arrayrulewidth\savedwidth}}


%
\usepackage{setspace}
%

\usepackage[dvipsnames]{xcolor}


%
\usepackage[hang,aboveskip=5pt,labelfont=bf,labelsep=period,justification=justified,singlelinecheck=off,textfont=small]{caption}
\renewcommand{\figurename}{Fig.}

%\bibliographystyle{naturemag}

\usepackage[numbers,sort&compress]{natbib}
\bibliographystyle{unsrt}
%
%

%
%\makeatletter
%\renewcommand{\@biblabel}[1]{\quad#1.}
%\makeatother

%
\date{}

%
\usepackage{lastpage,fancyhdr,graphicx}
\usepackage{epstopdf}
\pagestyle{myheadings}
\pagestyle{fancy}
\fancyhf{}
%
%
\rfoot{\thepage/\pageref{LastPage}}
\renewcommand{\footrule}{\hrule height 2pt \vspace{2mm}}
\fancyheadoffset[LR]{1.00in}
\fancyfootoffset[LR]{1.00in}
%

%


\renewcommand{\a}{{\bf a}}
\renewcommand{\b}{{\bf b}}

\newcommand{\ol}{\overline}
%
\newcommand{\ep}{\epsilon}
\newcommand{\s}{\sigma}
\renewcommand{\L}{{\cal L}}
\newcommand{\F}{{\cal F}}
\newcommand{\comment}[1]{{\noindent \color{blue} #1}}
\newcommand{\corr}[1]{{\noindent \color{red} \small #1}}
\newcommand{\add}[1]{{\noindent \color{green} \small #1}}
\newcommand{\purple}[1]{\textcolor{purple}{#1}}
\newcommand{\outcomment}[1]{}
%
\newcommand{\EQ}{\begin{equation}}
\newcommand{\EE}{\end{equation}}
\newcommand{\EQA}{\begin{eqnarray}}
\newcommand{\EEA}{\end{eqnarray}}
\newcommand{\productlog}{\mathrm{ProductLog}}

%
\graphicspath{ {../figures/} }
\newcounter{firstbib}
%


\begin{document}
	\vspace*{0.2in}

	%
	\begin{flushleft}
		{\Large
			\textbf\newline{Survival of the complex/Survival of the most flexible}
		}
		\newline
		\\
		Tom R\"oschinger\textsuperscript{1},
		Simone Pompei\textsuperscript{2},
		Michael L\"assig\textsuperscript{3,*},
		\\
		\bigskip
		\textbf{1} Division of Chemistry and Chemical Engineering, California Institute of Technology, Pasadena, CA 91125 \\
		\textbf{2} \\
		\textbf{3} Institut f\"ur Biologische Physik, Universit\"at zu K\"oln,
		Z\"ulpicherstr. 77, 50937, K\"oln, Germany
		\\
		\bigskip

		%
		%
		%
		%
		%
		%\textbf{\P} These authors contributed equally to this work.

		%
		* mlaessig@uni-koeln.de

	\end{flushleft}
	\subsection*{To Do List}
	
	\begin{itemize}
		\color{ForestGreen}
		\item Urgent:
		\begin{itemize}
			\item Michael: Supplementary Methods 3: Does the diversity within a population depend on the alphabet size, as seen from the analytic results? $\Delta E ~\sim \left(\frac{n}{n-1}\right)^2$. This is important for the results in this section. Either mathematical proof or logical explanation.
			\item Tom: Think about loss at various fitness parameters, and how that relates to observable non-equilibrium.
			\item Tom: Get a first draft of the methods section ready so it is easy to refer to it when writing the results. The methods should not change too much anymore, unless we include data analysis, but this can be added later.
			\item Tom: Add conceptual figures (Optimal length scaling with non-equilibrium, effect of ratchet, two dimensional distribution $Q(\gamma, l)$ moving in 2D space with increasing $\kappa$).
		\end{itemize}
		\item Remaining Tasks:
		\begin{itemize}
			\item Get all necessary simulations done and store results with README.md files to remember what they where.
			\item Finalize analytical computations in supplementary methods.
			\item Formulate punchline of the paper.
			\item Check references.
			\item Write introduction based on the results.
			\item Finalize results section.
			\item Spend \~ one month looking for possible data sets.
			\item Either add data analysis or write discussion and be done.
		\end{itemize}
	\end{itemize}
	
	\subsection*{Abstract}
	

	\subsection*{Introduction}
	The complexity of a system is a crucial component in molecular evolution. Evolution of gene regulation is often assumed to be driven by the gain and loss of transcription factor binding sites (TFBS), rather than the evolution of a transcription factor (TF) itself \cite{tugrul_dynamics_2015}. This assumption is based on the pleotropy of most TFs that have been observed, and that the pleotropic effects of a mutation in such a transcription factor are deleterious for most interactions, such that no change is possible. However, these TF's are conserved across species to begin with \cite{schmidt_five-vertebrate_2010}. Some classes of TFs display extensive levels of species diversity while maintaining structure and function \cite{nowick_lineage-specific_2010} and it has been discussed that protein evolution is a crucial source of developmental variation \cite{lynch_resurrecting_2008}, but the consequences of protein evolution remain unclear \cite{wagner_gene_2008, voordeckers_how_2015}. The evolution of gene regulatory networks is assumed to be driven by modifications 
	and the evolution of TFs \cite{lozada-chavez_bacterial_2006, perez_evolution_2009}.
	
	\subsection*{Methods}
	\subsubsection*{Fitness Model}
    We are considering systems where the fitness is proportional to the binding of a single molecule to its functional site, such as a transcription factor to its operator. In thermodynamic equilibrium, the binding probability for this system is \cite{lassig_biophysics_2007}
	\begin{equation}
		p_+(E)=\frac{1}{1+e^{\beta\,(E-F_0)}},
    \end{equation}
    where $F_0$ is the free energy of a random genome. This function has a sigmoid shape, which can approximated as an exponential function close to one of the plateaus. In analytical computations we often approximate high binding probability plateau as $p_+(E) \sim (1-e^{\beta\,(E-F_0)})$. In addition, we consider the fitness effect of the binding site length. Generally, we assume that longer binding sites come with an increased fitness cost, since genome size is under selection especially in prokaryotes (\purple{citation}). Therefore, we include a linear fitness cost $f_l$ per position of the binding site to the system,
    \begin{equation}
        F(E,l) = f_0p_+(E,l)-f_l\,l,\label{equ:fitness_function}
    \end{equation}
    where $f_0$ is the proportionality factor between binding probability to fitness. Note that the linear fitness cost for binding site length does not influence the dynamics of binding sites of fixed length, since the term maintains constant and selection coefficients are calculated as fitness differences, therefore canceling any constant additional term.

\subsubsection*{Binding Energy Model}
    We assume a minimal energy model, where each position contributes independently to the total binding energy of the sequence, which is called the independent nucleotide approximation and commonly used for Protein-DNA interactions \cite{stormo_specificity_1998,djordjevic_selex_2007}. Therefore, we assume that minimal binding energy is achieved by a reference sequence, and each mismatch from that sequence brings a fixed cost to the binding energy of about $\epsilon\beta\approx 2-3$ \cite{lassig_biophysics_2007}. In this work we fix the energy cost per mismatch to be $\epsilon\beta=2$, independent of the actual nucleotide at the position. For specific examples, there are methods to obtain real energy matrices that make it possible to compute the actual binding energy of a transcription factor to its binding site \cite{barnes_mapping_2019,ireland_deciphering_2020}.\\
    Due to the linearity of the model, the total binding energy will increase with the length of the binding site, for both unspecific and specific sequences. We assume that the total number of unspecific binding site maintains constant, hence, the free energy difference $\Delta E$ that is required to acquire specificity compared to off target binding sites maintains constant as well,
    \begin{equation}
		\Delta E  = E_0(l) - E^*(l),
	\end{equation}
    where we relabeled the free energy threshold in the sigmoid fitness landscape as $E^*(l)$ and $E_0(l) = 3/4 l\epsilon$, since a random sequence has $1/4\, l$ just by chance. The free energy threshold is $\Delta E=10 k_B T$ (\purple{argue either from minimal length of binding site or other source}).
    

	\subsubsection*{Genetic Load and Length}
	One measure of adaption is the genetic load, which is computed as the difference of the mean fitness of a population to the maximal achievable fitness. Using the exponential approximation of the upper plateau of the binding term in the fitness landscape, the genetic load is simply given as the derivative of the fitness landscape evaluated at the mean energy $\mathcal{L} = \beta |f'(\Gamma)|$ \cite{held_survival_2019}. The time evolution of the trait mean is given by the stochastic equation
	\begin{equation}
		\dot{\Gamma}=m^\Gamma+\Delta_E f'(\Gamma)+\chi_\Gamma(t),
    \end{equation}
    where $m^\Gamma$ is the mutational drift, $\Delta_E$ the diversity within a population regarding binding energy, and $\chi_\Gamma(t)$ describes stochastic fluctuations. 
	\subsubsection*{Mutation model}

	

	\subsection*{Results}
	\subsubsection*{Evolutionary Steady State in Non-equilibrium}
	{\color{ForestGreen} This section should contain all necessary information and needs work on the language.}\\
	We start by investigating how externally driven mutations that change the consensus sequence of the binding site force the system into a new state that is different from the steady state that can be observed when binding is only disrupted by mutations in the binding site. Therefore, we use the Fokker-Planck equation describing the evolution of the mean $\Gamma$ of a quantitative trait $E$ \cite{nourmohammad_evolution_2013},
	\begin{equation}
		\frac{\partial}{\partial t}Q(\Gamma,t)=\left[\frac{\tilde{g}^{\Gamma\Gamma}}{2N}\frac{\partial^2}{\partial\Gamma^2}- \frac{\partial}{\partial\Gamma}\left(m^\Gamma+\tilde{g}^{\Gamma\Gamma}\tilde{s}_\Gamma\right)\right]Q(\Gamma,t)\label{equ:trait_diffusion}.
	\end{equation}
	The equation can be broken down into three evolutionary forces: genetic drift, selection and mutations. In the low mutation rate regime, the selection term can be written as $\tilde{s}_\Gamma = \partial_\Gamma f(\Gamma)$, i.e., the derivative of the fitness landscape at the value of the trait mean. The effective diffusion coefficient is given by $\tilde g^{\Gamma\Gamma} = \langle \Delta\rangle \sim \mu$ (\purple{include scaling with alphabet size here?}), hence scaling the strength of selection with the variation of the trait within a population (\purple{mention Fisher's theorem here?}). 
	The mutational drift term is $m^\Gamma = -\frac{n}{n-1} (\Gamma-\Gamma_0)$, where $n$ is the alphabet size of the binding site and $\Gamma_0$ is the mean trait in the absence of selection. 
	The mutation term describes deterministic drift that sites undergo due to mutations. If this term dominates the deterministic behavior, then most sites will have binding energies similar to random sequences. Driver mutations have the same impact on the binding energy as trailer mutations. Hence, in the evolutionary description of traits, we can add the driver mutation rate $\rho$ to the trailer mutation rate $\mu$ in the mutation term, $m^\Gamma\sim-(\rho+\mu)(\Gamma-\Gamma_0)$, for details see \textcolor{blue}{methods oder supplementary}.\\
	The selection term $\tilde{g}^{\Gamma\Gamma}\tilde{s}_\Gamma$ describes the selection coefficient of the trait mean value $\tilde{s}_\Gamma$ and how it is amplified by the diversity within a population $\tilde{g}^{\Gamma\Gamma}=\langle\Delta_E\rangle $. For low mutation rates, the diversity within a population is given by its value in the absence of selection, and very small. Most positions are monomorphic, thus a mutation in the driver sequence changes the binding energy for nearly all sites equally. Therefore, the diversity within a population is not affected by driver mutations, and the selection term is invariant.\\
	The non-deterministic term $\tilde{g}^{\Gamma\Gamma}/2N$ describes how mutations lead to a spread in mean binding energies across populations. In the original formulation this term also scales with the mean diversity within a population. However, driver mutations also lead to an increased diversity across populations, since they change the mean binding energy, while not effecting the mean diversity within a population. Hence, we have to distinguish between the two response terms $\tilde{g}^{\Gamma\Gamma}$. The response coefficient in the selection term, which only scales with the trailer mutation rate $\mu$, is labeled $\tilde{g}^{\Gamma\Gamma}_s\sim\mu$. The response coefficient in the non-deterministic term is scaling with both mutation rates and is labeled $\tilde{g}^{\Gamma\Gamma}_d\sim\mu+\rho$.\\
	In the absence of selection, the steady state distribution $Q_0(\Gamma)$ is the same as in the absence of driving, since the scaling of both terms cancels out in the final solution (\textcolor{blue}{Methods}). However, in a driven environment $\rho>0$, the steady state solution becomes
	\begin{equation}
		Q(\Gamma)=\frac{1}{Z}Q_0(\Gamma)\exp\left[\frac{2N}{1+\kappa}F(\Gamma)\right],\label{equ:steady_state}
	\end{equation}
	where $\kappa=\rho/\mu$ is the ration of driver and trailer mutation rate, which is the key parameter describing the strength of non-equilibrium. A rescaling of the fitness landscape $\hat{F}(\Gamma)=F(\Gamma)/(1+\kappa)$ reproduces the steady state distribution in equilibrium with a weaker fitness landscape. This result is explained by the different scalings of the population genetic terms with non-equilibrium. The mutation drift towards smaller binding energies increases as well as diffusion across populations, while selection is not increased and therefore relatively weaker.\\
	Note, that the rescaling of the fitness landscape effectively only applies to the term in the fitness landscape regarding the binding probability. The fitness length cost is absorbed in the normalization of the steady state distribution in \ref{equ:steady_state}.
	In the low mutation rate limit, an exact steady state distribution can be numerically computed from substitution rates (Methods or Supplementary). In the regime of small selection coefficients $Ns\sim1$, an analytic steady state can be approximated (Methods or Supplementary), which is equal to the result in equation \ref{equ:steady_state}.

	\begin{figure}
		\includegraphics[width=\linewidth]{figure1_concept.pdf}\caption{\color{ForestGreen} Conceptual figure. Left side should show various distributions that evolved for fixed lengths at different fitness parameters $f_0$ and how they change with increasing $\kappa$. For now, I simply plotted histograms, and the colors in each plot are the different fitness parameters (as seen in the legend on the right plot). Then, to show the distributions vary, we could show the mean (or peak) of each distribution on the right side and how it changes with increasing $\kappa$. This simulation was run \textbf{without recovery} of sites at the lower plateau, but is being repeated right now with recovery. This figure would support the first claim that the distribution in non-equilibrium can be written as the steady-state distribution with and additional factor of $1/(1+\kappa)$ in the exponent. Axis labels are going to be fixed in the next iteration.}\label{fig1}
	\end{figure}


	\subsubsection*{Optimal Binding Site Length}
	{\color{ForestGreen} This section needs to be brought on par with the methods section. The part about genetic load from Torsten/Daniel paper should be in the methods, and everything on top of that should be described in this section. Discuss possible regimes of $\kappa$ in biological systems, but maybe this should be done in a different section.}\\
	First, we compute the binding site length which minimizes the genetic load, therefore being the fittest and is expected to be the steady state that binding site lengths are evolving towards.
	Non-equilibrium weakens the ability to find specific binding sites, while the length fitness cost is independent of the state of the binding. Hence, the fitness trade-off changes with the level of non-equilibrium. We can consider the fitness effects in the terms of genetic load. For weak non-equilibrium most sites are functional and are at the edge of the upper fitness plateau, which can approximated by an exponential tail. Then the genetic load of a population is given by $\mathcal{L}\sim \left|\frac{\partial}{\partial \Gamma} f(\Gamma,l)\right|+f_l\,l$. The average load across populations is given by the load at the deterministic mutation-selection balance $\hat{\Gamma}$(Methods or Supplementary) (\textcolor{blue}{Cite Torsten and Daniel Paper}). We retrieve two terms for the genetic load, with different scalings of the binding length. The linear fitness cost for binding length, and the binding probability term with scales inversely with binding length,
	\begin{equation}
		\mathcal{L} = \xi \frac{l_0}{l}(1+\kappa)+\lambda \, \frac{l}{l_0} = \mathcal{L}_\kappa + \mathcal{L}_\lambda,
		\label{equ:maintext_load}
	\end{equation}
	Where $\xi$ is a combination of various parameters, $\lambda=2Nf_l/l_0\sim 1$ is the scaled fitness cost for length,  and $l_0$ is a length scaling parameter (supplementary methods). A derivation of this equation can be found in the supplementary methods. Due to the trade off between the cost of adding positions to the binding site and reducing load by adding positions, there is an optimal binding site length, which minimizes the total genetic load,
	\begin{equation}
		l_\text{opt}\sim l_0\sqrt{\frac{1+\kappa}{\lambda}}.
	\end{equation}
	In equilibrium, the optimal binding site length is close to the length scaling parameter, $l\sim l_0$.  


\subsubsection*{Dynamical Evolution of Binding Site Length}
{\color{ForestGreen} This section should be very descriptive, since it is one of the main results. And why considering the ratchet is important when writing down models of this kind which could consider binding site length. Also, could this be a reason for larger genomes, as soon as selection pressure decreases? Also, this section should have a minimal number of equations, just tell the story, and any computation should be in the supplementary methods, to make this section as straightforward to understand as possible. }\\
In a real population, length adaptation is a dynamical process, driven by mutations which include a new position in the sequence in its recognition. We assume these mutations to be very rare compared to other type of mutations, such that the timescales separate. Therefore, we can assume that the system is in the respective steady state of binding energies each time a length mutation occurs. Assuming an alphabet size of 4 and using the match/mismatch energy model (see Methods), a newly added position will be a match with probability 1/4, since it has not been selected for yet and hence is random compared to the driver sequence. Length increase mutations are on average neutral in respect to the adaption term in the fitness landscape in first order, and are only influenced by the fitness cost of length (supplementary methods),
\begin{equation}
	\sigma^+ = -\frac{\lambda}{l_0} =-\frac{\partial \mathcal{L}_\lambda}{\partial l}.
\end{equation}
Since $\lambda\sim 1$, length increase mutations are in the regime of weak selection (\purple{Here either an explanation of what this means or a nice reference}).
However, length decrease mutations are affected by the fact that the binding site is adapted, i.e., removing a position is likely to hit an interaction which is beneficial to the binding interaction. If the binding site has $k$ matches, then the probability of removing a match is $k/l$, which is much larger than the probability of adding a match, which we showed to be $1/4$. The selection coefficient for removing a position has an additional term,
\begin{equation}
	\sigma^- = \frac{\lambda}{l_0} - \xi \epsilon\beta \left(\frac{l_0}{l}\right)^2 (1+\kappa) = \frac{\partial \mathcal{L}_\lambda}{\partial l} + l_0 \epsilon\beta\frac{\partial \mathcal{L_\kappa}}{\partial l}.
\end{equation}
Unexpectedly, the adaptation term is scaled by the effective length scaling parameter, and the mutational effect. Therefore, the selection coefficient for length decrease mutations shows marginal selection, $\sigma^-\sim (l/l_0)^2\sim 1$. Due to the separation of time scales, we can write down substitution rates for length mutations, which lead to a steady state distribution. In first order, we can compute the substitution rates as 
\begin{equation}
	u_{+/-} \sim 1 + \frac{\sigma^{+/-}}{2} \sim \exp\left[\frac{\sigma^{+/-}}{2} \right].
\end{equation}
Using detailed balance (\purple{Maybe avoid this term here?}), we can compute the steady state distribution for binding site lengths. Using the rates above, we find that the resulting distribution is
\begin{equation}
	P(l) \simeq \exp\left[-\frac{\epsilon\beta l_0}{2}\mathcal{L}_\kappa - \mathcal{L}_\lambda\right].
	\label{equ:length_distribution}
\end{equation}
When comparing the maximum of this distribution with the length with minimizes the load, we find that the length $l_opt^*$ which optimizes the effective potential resulting from the dynamical calculation is larger by a factor $\sqrt{l_0}$,
\begin{equation}
	l_{opt}^* \simeq \sqrt{l_0}\,l_{opt}.
\end{equation}
This is an enormous difference, see Figure~(\purple{insert figure}), and shows how important it is to consider the dynamics of the evolution of binding site lengths. Transcription factor binding sites have lengths around 10bp in prokaryotes (\purple{cite "Why are TF binding sites 10bp long."}), so we can tune the fitness cost for length $f_l$ such that the optimal length computed from the dynamical computation is around that value in equilibrium ($\kappa=0$). Then, we observe an increase in binding site length with increasing non-equilibrium, due to the sites being under more selective pressure to maintain their binding specificity, which shifts the weights in the trade off with the cost of increased binding site length.

\subsubsection*{Data}
\textcolor{ForestGreen}{Placeholder for a possible section on some data. This has the lowest priority so far, and I think we should finish everything but the discussion before we dive into this.}

\comment{
\paragraph{Comment 29. 7., edited 8. 9. Does the slow dynamics of $\ell$ lead to a steady state that is localized at $\ell_{\rm opt}$?}
\begin{enumerate}
\item Our model has a conditional stationary distribution $Q(k|\ell)$ peaked at $k/\ell \equiv \gamma$ with
\EQ
\gamma - \gamma_0 = \frac{\ell_0}{\ell}
\EE
and $\gamma_0 = 1/4$. This stationary state has a scaled load $\L \equiv (F - F_{\max})/ \tilde \sigma$ given by
\EQ
\L = \L_\kappa + \L_\lambda = (\gamma - \gamma_0) (1 + \kappa) + \lambda  \frac{\ell}{\ell_0},
\EE
leading to an optimum length
\EQ
\ell_{\rm opt} = \ell_0 \sqrt{\frac{1+\kappa}{\lambda}}.
\EE

\item In the exponential regime of the (scaled) fitness landscape $\F(k |\ell) \equiv F(k|\ell) / \tilde \sigma$, which is relevant for stable sites, the scaled load equals the scaled selection coefficient of site mutations,
\EQ
s = \frac{\partial \F(k,\ell)}{\partial k} = \L_\kappa = (\gamma - \gamma_0) (1 + \kappa);
\EE
prefactors to be double-checked with Torsten-Daniel paper.

\item Perturbation theory in $\ep \equiv 1/\ell_0$. We can distinguish terms of order $\ep$ and of order $\ell/\ell_0 \sim 1$. We have $\gamma \sim \lambda \sim 1$ and $\L \sim 1$; the scaling of $\lambda$ follows from the requirement $\ell_{\rm opt} \approx \ell_0$ at $\kappa = 0$. In particular, we can distinguish selection coefficients $ \sim 1/\ell_0$ (weak selection) and $ \sim 1$ (marginal selection).

\item Adiabatic substitution dynamics of $\ell$. Length changes are assumed to occur with much smaller rates than mutations; we assume they are emitted from the steady state of the $k$ dynamics. Each length change is coupled to a stochastic change of $k$. Because the dynamics takes place in the regime of weak and marginal selection, we use a linear approximation for substitution probabilities. Hence, we can first compute average selection coefficients $\bar s_+$ and $\bar s_-$ for changes of $\ell$ and then evaluate the corresponding substitution rates using these averages.

In this framework, we find the processes $\ell \to \ell +1$ with average selection coefficient
\EQA
\bar s_+ & = & \frac{3}{4} \frac{(-s)}{4} + \frac{1}{4} \frac{3s}{4} - \frac{\lambda}{\ell_0}
\\
& = & - \frac{\lambda}{\ell_0}
\EEA
(near-neutral evolution) and the processes $\ell \to \ell - 1$ with average scaled selection coefficent
\EQA
\bar s_- & = & \gamma \frac{(-3s)}{4} + (1 - \gamma) \frac{s}{4} + \frac{\lambda}{\ell_0}
\\
& = & -(\gamma - \gamma_0) s + \frac{\lambda}{\ell_0}
\\
& = & - (\gamma - \gamma_0)^2 (1 + \kappa) + \frac{\lambda}{\ell_0}
\\
& = & \ell_0 \frac{\partial \L_\kappa}{\partial \ell} + \frac{\partial \L_\lambda}{\partial \ell}
\EEA
(marginal selection). These selection coefficients are of different magnitude: length increases are constrained only by the weak linear term, length decreases are marginally constrained by conservation of site function.

\item These asymmetric dynamics define an adaptive ratchet, which acts to increase selection for complexity. A typical cycle $\ell \to \ell + 1 \to \ell$ has the following average scaled fitness balance:
\begin{itemize}
\item substitution $\ell \to \ell + 1$: weak fitness loss, $\bar s_+ = - \lambda / \ell_0 = - O(\ep)$;
\item equilibration at $\ell + 1$: weak fitness gain, $\Delta \L_{\ell + 1} = \L_\kappa (\ell) - \L_\kappa (\ell +1) \sim \ell_0 /\ell^2 = O( \ep)$;
\item substitution $\ell + 1 \to \ell$: marginal fitness loss,   $\bar s_- = - \ell_0^2 /\ell^2 =- O(\ep^0) $;
\item equilibration at $\ell$: marginal fitness gain, $\Delta \L_{\ell} = - \L_\kappa (\ell) + \L_\kappa (\ell +1) - \bar s_- = O(\ep^0)  $.
\end{itemize}
This cycle could inform a Figure.

\item Effective fitness landscape for length changes. We compute the ratio of the forward and backward substitution rate in first-order approximation,
\EQA
\frac{u_{\ell \to \ell+1}}{u_{\ell + 1 \to \ell}} & = & \frac{\exp(\bar s_+/2)}{\exp(\bar s_-/2)} +O(s^2)
\\
& \simeq & \exp \left [ - \frac{\ell_0}{2} (\L_\kappa (\ell+1) - \L_\kappa (\ell)) - (\L_\lambda (\ell+1) - \L_\lambda (\ell)) \right ]
\\
& \equiv & \exp \big [ \F_{\rm eff} (\ell+1) - \F_{\rm eff} (\ell) \big ]
\label{u_ratio}
\EEA
with an effective scaled fitness potential
\EQ
\F_{\rm eff} (\ell) = - \frac{\ell_0}{2} \L_\kappa (\ell) - L_\lambda (\ell) .
 \EE
These rates define an equilibrium distribution
\EQ
Q(\ell) \sim \exp [ \F_{\rm eff} (\ell)],
\EE
which is peaked at
\EQ
\ell^* = \ell_0 \sqrt{ \frac{\ell_0}{2} \frac{(1+\kappa)}{\lambda}} = \ell_{\rm opt} \sqrt{ \frac{\ell_0}{2}} .
\EE
Compared to the naive distribution $Q(\ell) \sim \exp [ - \L (\ell)]$, which would lead to a peak at $\ell_{\rm opt}$, the ratchet mechanism enhances the evolution of complexity.

\item Summary.
\begin{itemize}
\item Non-equilibrium weakens the selection on point mutations at a given complexity, as expressed by an effective fitness landscape
\EQ
\F_{\rm eff} (k|\ell) \sim \frac{1}{1 + \kappa} .
\EE
\item Non-equilibrium introduces ratchet selection for complexity, as expressed by an effective fitness landscape
\EQ
\F_{\kappa, \rm eff} (\ell) \sim \ell_0(1+ \kappa).
\EE
This landscape is proportional to $\ell_0$; that is, complexity begets more complexity.
\end{itemize}

\item Questions:\\
(1) Is this in qualitative accordance with the evaluation of the rates in (\ref{u_ratio}) beyond first order?
\\
(2) How does the fitness balance compare with the numerics?
\\
(3) How does $\ell^*$ compare with the numerics?
\end{enumerate}
}



\subsection*{Discussion}
\textcolor{ForestGreen}{Here we should discuss the meaning of our results. The two factors of increased binding site length. First, the dynamics of length evolution lead to a much higher binding site length in equilibrium, than observed from minimizing the genetic load. But. Also we should discuss which data could support our theory. And finally, the shortcoming of the theory. Like how the approximation of the distribution is not really good for higher selection coefficients. And what how it could be used to improve. And finally how we think experiments could be set up to test the hypthesis experimentally.}



\bibliography{complexity_refs.bib}




%\begin{thebibliography}{10}
%
%	\expandafter\ifx\csname url\endcsname\relax
%	\def\url#1{\texttt{#1}}\fi
%	\expandafter\ifx\csname urlprefix\endcsname\relax\def\urlprefix{URL }\fi
%	\providecommand{\bibinfo}[2]{#2}
%	\providecommand{\eprint}[2][]{\url{#2}}
%
%
%	\bibitem{dynamics2015}
	\bibinfo{author}{Tuğrul, M.}, \bibinfo{author}{Paixão, T.},
	\bibinfo{author}{Barton, Nicholas~H.} \& \bibinfo{author}{Tkačik, G.}
	\newblock \bibinfo{title} {Dynamics of Transcription Factor Binding Site Evolution}.
	\newblock\emph{\bibinfo{journal}{PLOS Genetics}}
	\textbf{\bibinfo{volume}{11}}
	\bibinfo{pages}{1-28} (\bibinfo{year}{2015})
	
	\bibitem{schmidt2010five}
	\bibinfo{author}{Schmidt, D.} \& \bibinfo{author}{et.al.}
	\newblock\bibinfo{title}{Five-vertebrate ChIP-seq reveals the evolutionary dynamics of transcription factor binding}.
	\newblock\emph{\bibinfo{journal}{Science}}
	\textbf{\bibinfo{volume}{328}}
	\bibinfo{pages}{1036--1040} (\bibinfo{year}{2010})
	
	\bibitem{Nowick2010}
	\bibinfo{author}{Nowick, K.} \& \bibinfo{author}{Stubbs, L.}
	\newblock\bibinfo{title}{Lineage-specific transcription factors and the evolution of gene regulatory networks}.
	\newblock\emph{\bibinfo{journal}{Briefings in Functional Genomics}}
	\textbf{\bibinfo{volume}{9}}
	\bibinfo{pages}{65--78} 
	(\bibinfo{year}{2010})
	
	\bibitem{lynch2008resurrecting}
	\bibinfo{author}{Lynch,V.~J.} \& \bibinfo{author}{Wagner, G.~P.}
	\newblock\bibinfo{title}{Resurrecting the role of transcription factor change in developmental evolution}.
	\newblock\emph{\bibinfo{journal}{Evolution: International Journal of Organic Evolution}}
	\textbf{\bibinfo{volume}{62}}
	\bibinfo{pages}{2131--2154} (\bibinfo{year}{2008})
	
	\bibitem{WAGNER2008377}
	\bibinfo{author}{Wagner, G.~P.} \& \bibinfo{author}{Lynch,V.~J.}
	\newblock\bibinfo{title}{The gene regulatory logic of transcription factor evolution}.
	\newblock\emph{\bibinfo{journal}{Trends in Ecology \& Evolution}}
	\textbf{\bibinfo{volume}{23}}
	\bibinfo{pages}{377 - 385} (\bibinfo{year}{2008})
	
	\bibitem{Voordeckers2015}
	\bibinfo{author}{Voordeckers, K.}, \bibinfo{author}{Pougach, K.} \& \bibinfo{author}{Verstrepen, K.~J.}
	\newblock\bibinfo{title}{How do regulatory networks evolve and expand throughout evolution?}.
	\newblock\emph{\bibinfo{journal}{Current Opinion in Biotechnology}}
	\textbf{\bibinfo{volume}{34}}
	\bibinfo{pages}{180--188} 
	(\bibinfo{year}{2015})
	
	\bibitem{chavez2006}
	\bibinfo{author}{Lozada-Chávez, I.}, \bibinfo{author}{Janga, S.~C.} \& \bibinfo{author}{Collado-Vides, J.}
	\newblock\bibinfo{title}{Bacterial regulatory networks are extremely flexible in evolution}.
	\newblock\emph{\bibinfo{journal}{Nucleic Acids Research}}
	\textbf{\bibinfo{volume}{34}}
	\bibinfo{pages}{3434--3445} 
	(\bibinfo{year}{2006})
	
	\bibitem{perez2009}
	\bibinfo{author}{Perez, J.~C.} \& \bibinfo{author}{Groisman, E.~A.}
	\newblock\bibinfo{title}{Evolution of Transcriptional Regulatory Circuits in Bacteria,
		Cell}.
	\newblock\emph{\bibinfo{journal}{Cell}}
	\textbf{\bibinfo{volume}{138}}
	\bibinfo{pages}{233--244} 
	(\bibinfo{year}{2009})
	

	
	\bibitem{lassig2007}
	\bibinfo{author}{L\"assig, M.}
	\newblock\bibinfo{title}{From biophysics to evolutionary genetics: statistical aspects of gene regulation}.
	\newblock\emph{\bibinfo{journal}{BMC bioinformatics} }\textbf{\bibinfo{volume}{8}},
	\bibinfo{pages}{1} (\bibinfo{year}{2007}).
	
	
	\bibitem{MADANBABU2006614}
	\bibinfo{author}{Babu, M.~M.}, \bibinfo{author}{Teichmann, S.~A.} \& \bibinfo{author}{Aravind, L.}
	\newblock\bibinfo{title}{Evolutionary Dynamics of Prokaryotic Transcriptional Regulatory Networks}.
	\newblock\emph{\bibinfo{journal}{Journal of Molecular Biology}}
	\textbf{\bibinfo{volume}{358}}
	\bibinfo{pages}{614 - 633} 
	(\bibinfo{year}{2006}).
	
	\bibitem{nourmohammad2013evolution}
	\bibinfo{author}{Nourmohammad, A.}, \bibinfo{author}{Schiffels, S.} \&
	\bibinfo{author}{Lässig, M.}
	\newblock \bibinfo{title}{Evolution of molecular phenotypes under stabilizing selection}.
	\newblock \emph{\bibinfo{journal}{Journal of Statistical Mechanics: Theory and Experiment}}
	\textbf{\bibinfo{volume}{2013}},
	\bibinfo{pages}{P01012} (\bibinfo{year}{2013}).
	
	\bibitem{mustonen2010fitness}
	\bibinfo{author}{Mustonen, V.} \&
	\bibinfo{author}{Lässig, M.}
	\newblock \bibinfo{title}{Fitness flux and ubiquity of adaptive evolution}.
	\newblock \emph{\bibinfo{journal}{Proceedings of the National Academy of Sciences}}
	\textbf{\bibinfo{volume}{107}},
	\bibinfo{pages}{4248--4253} (\bibinfo{year}{2010}).
	
	\bibitem{berg1987selection}
	\bibinfo{author}{Berg, Otto~G.} \&
	\bibinfo{author}{von Hippel, Peter~H.}
	\newblock \bibinfo{title}{Selection of DNA binding sites by regulatory proteins: Statistical-mechanical theory and application to operators and promoters}.
	\newblock \emph{\bibinfo{journal}{Journal of molecular biology}}
	\textbf{\bibinfo{volume}{193}},
	\bibinfo{pages}{723--743} (\bibinfo{year}{1987}).
	
	\bibitem{kimura1962probability}
	\bibinfo{author}{Kimura, M.}
	\newblock \bibinfo{title}{On the probability of fixation of mutant genes in a population}.
	\newblock \emph{\bibinfo{journal}{Genetics}}
	\textbf{\bibinfo{volume}{47}},
	\bibinfo{pages}{713} (\bibinfo{year}{1962}).
	
	\bibitem{berg2003stochastic}
	\bibinfo{author}{Berg, J.} \&
	\bibinfo{author}{Lässig, M.}
	\newblock \bibinfo{title}{Stochastic evolution of transcription factor binding sites}.
	\newblock \emph{\bibinfo{journal}{Biophysics}}
	\textbf{\bibinfo{volume}{48}},
	\bibinfo{pages}{36--44} (\bibinfo{year}{2003}).
	
	\bibitem{stormo1998specificity}
	\bibinfo{author}{Stormo, Gary~D.} \&
	\bibinfo{author}{Fields, Dana~S.}
	\newblock \bibinfo{title}{Specificity, free energy and information content in protein--DNA interactions}.
	\newblock \emph{\bibinfo{journal}{Trends in biochemical sciences}}
	\textbf{\bibinfo{volume}{23}},
	\bibinfo{pages}{109--113} (\bibinfo{year}{1998}).
	
	\bibitem{djordjevic2007selex}
	\bibinfo{author}{Djordjevic, M.}
	\newblock\bibinfo{title}{SELEX experiments: new prospects, applications and data analysis in inferring regulatory pathways}.
	\newblock\emph{\bibinfo{journal}{Biomolecular engineering}}
	\textbf{\bibinfo{volume}{24}}
	\bibinfo{pages}{179--189} (\bibinfo{year}{2007}).
		
	\bibitem{stormo2010determining}
	\bibinfo{author}{Stormo, Gary D.} \& \bibinfo{author}{Stekel, Dov J.}
	\newblock\bibinfo{title}{Determining the specificity of protein--DNA interactions}.
	\newblock\emph{\bibinfo{journal}{Nature Reviews Genetics}}
	\textbf{\bibinfo{volume}{11}}
	\bibinfo{pages}{751} (\bibinfo{year}{2010})
	
	
	
	\bibitem{held2019}
	\bibinfo{author}{Held, T.}, \bibinfo{author}{Klemmer, T.},
	\& \bibinfo{author}{L\"assig, M.}
	\newblock \bibinfo{title} {Survival of the Simplest in Microbial Evolution.}.
	\newblock\emph{\bibinfo{journal}{Nature Communications}}
	\textbf{\bibinfo{volume}{10}}
	(\bibinfo{year}{2019})
%
%
%\end{thebibliography}



\section*{Acknowledgments}



\section*{Author Contributions}


\section*{Competing Interests}
The authors declare no competing interests.


\section*{Supplementary Information}

\paragraph*{Supplementary Methods~1}
\label{supp1}
{\bf Steady State Approximations.}\\

\noindent In the regime of low mutation rates, $\mu N\ll1$, a population is in a monomorphic state, i.e., dominated by a single genotype, 
for most of the time. A new mutation fixes in the population with probability $p(s)$ given by \cite{kimura_probability_1962},
\begin{equation}
p(s) = \frac{1-e^{-2s}}{1-e^{-2Ns}},
\end{equation}
where $s$ is the selection coefficient of the mutation and $N$ the effective population size. Then, substitutions occur at a rate
\begin{equation}
u(s)_{E\rightarrow E'} = N\mu_{E\rightarrow E'}p(s),
\end{equation}
where $\mu_{E\rightarrow E'}$ is the rate at which a mutation occurs which changing the binding energy from $E$ to $E'$. Here, we assumed that the fixation 
of a mutation is under selective pressure, which is the case for mutations in the trailer. We investigate the effects of mutations in the driver, which occur 
with rate $\rho$. These mutations are results of external events, e.g., a mutation in the coding sequence of a transcription factor leading to a missense mutation. 
In isolated system of driver and trailer, these mutations are not under selection pressure, but can change the binding energy nonetheless. 
Here we assume that these mutations happen for all driver sequences at the same time and therefore can be treated as substitutions as well. 
Thus, we add a term to the mutation rate in the form of
\begin{equation}
u(s)_{E\rightarrow E'}=   N\mu_{E\rightarrow E'}p(s) + \rho_{E\rightarrow E'},
\end{equation}
where $\mu_{E\rightarrow E'}$ is the mutation rate with driver mutations change the binding energy from $E$ to $E'$.
Assume, both trailer and driver have the same alphabet size, then both mutation rates are given by a constant rate times a factor accounting for 
the entropic difference between both states, $\mu_{E\rightarrow E'}=\mu\,w_{E\rightarrow E'}$ and $\rho_{E\rightarrow E'}=\rho\,w_{E\rightarrow E'}$. 
Even if the assumption of equal alphabet sizes doesn't hold, e.g., in protein-DNA interaction, the differences in alphabet sizes can be accounted for 
by rescaling the mutation rates by a constant factor. The entropic term $w$ is determined by the substitution rates in the absence of selection.
We are looking for the distribution of binding energies at steady state, which is equivalent to 
\begin{equation}
\frac{u(0)_{E\rightarrow E'}}{u(0)_{E'\rightarrow E}} = \frac{\mu_{E\rightarrow E'}}{\mu_{E'\rightarrow E}}= \frac{w_{E\rightarrow E'}}{w_{E'\rightarrow E}} = 
\frac{Q_0(E')}{Q_0(E)},
\end{equation}
where $Q_0(E)$ is the distribution of binding energies in the absence of fitness. Even though there is non-equilibrium on the level of genotypes, 
we can look for a steady state distribution of the binding energies. Since the dynamics are one dimensional and fully described by the substitution rates, 
we can find the steady state distribution $Q(k)$ by imposing detailed balance and solving
\begin{equation}
\frac{Q(E')}{Q(E)}=\frac{u(s)_{E\rightarrow E'}}{u(-s)_{E'\rightarrow E}}=  \frac{ N\mu_{E\rightarrow E'}p(s)+\rho_{E\rightarrow E'}}{N\mu_{E\rightarrow E'}p(s)
+\rho_{E\rightarrow E'}}.
\end{equation}
In equilibrium, $\rho=0$, the fraction has an exact solution \cite{berg_adaptive_2004},
\begin{equation}
	Q(k) = Q_0(k)\exp[2NF(k)].
	\label{equ:sub_equ_dist}
\end{equation}
In non-equilibrium $\rho>0$, the fraction does not simplify to an exact term as equation~\ref{equ:sub_equ_dist}, but we can compute the exact steady state 
distribution numerically. However, in the limit of small selection coefficients $Ns\sim1$, the can expand the fraction in orders of the selection coefficient 
and retrieve,
\begin{align}
\frac{Q_0(E')}{Q_0(E)}\frac{ N\mu p(s)+\rho}{N\mu p(s)+\rho}&\approx \frac{Q_0(E')}{Q_0(E)}\left[1+\frac{2s(N-1)}{1+\frac{\rho}{\mu}}+\frac{2s^2(N-1)^2}
{(1+\frac{\rho}{\mu})^2}+\mathcal{O}(s^3)\right] ,\nonumber\\
&\approx \frac{Q_0(E')}{Q_0(E)}\exp\left[\frac{2N}{1+\kappa}F(E)\right].
\end{align}
Populations are usually large, so $N-1\approx N$, and up to second order the series is equal to an exponential function. The resulting distribution is therefore 
given by
\begin{equation}
	Q(E) = Q_0(E) \exp\left[ \frac{2N}{1+\kappa} F(E) \right].
\end{equation}

\outcomment{
\paragraph*{Supplementary Methods~2}
\label{suppx}
{\bf Steady State Approximation using substitution rates.}\\ \\
In the low mutation rate limit, the mean binding energy is more discrete, and a substitution leads to bigger changes in the binding energy. Therefore, selection coefficients can get larger than $N\,s\sim1$, and the approximation of the steady state deviates from the exact distribution. Especially for short binding sites, the range of possible mutations with small effects decreases to the order of 1. To correct for the discrete energy space, we can rescale the non-equilibrium parameter $\kappa$ to an effective parameter $\tilde{\kappa} = \xi\kappa$, where $\xi$ is the rescaling factor which reproduces the first moment of the exact distribution the best. $\alpha$ takes maximal values of order 1, and in figure \ref{supfig1} the retrieved value is shown for different binding lengths $l$ and different fitness parameters $f_0$. For longer binding sites or smaller total selection coefficients, the rescaling factor approaches 1, which is consistent with the fact, that longer binding sites have more mutations with small selection coefficients and approach the continuum limit.

Due to the discrete mean binding energies in the low mutation rate limit, selection coefficients can become larger than in the continuous regime. Therefore the approximation deviates from the exact numerical distribution when the discretization becomes a big factor, i.e., for short sequences or large selections coefficients. When the approximation deviates from the exact solution, it underestimates the effect of non-equilibrium.
%\begin{figure}
%	\includegraphics[width=0.5\linewidth]{robustness.pdf}
%%	\caption{\textbf{Robustness in discrete trait space.} }\label{supfig1}
%\end{figure}
}

\clearpage
\paragraph*{Supplementary Methods~2}
\label{supp2}
\textbf{Scaling of Mutational Drift and Population Diversity with Alphabet Size and Driver mutations.}\\

\noindent Here we use the population genetic notation of quantitative traits \cite{nourmohammad_evolution_2013}, and give a general form of the mutation drift. In addition we compute the contribution of driver mutations to the mutation drift.\\
The frequency of nucleotide $i$ at locus $k$ is given by $y_k^i$. The mutation drift of trailer mutations is therefore given by
\begin{equation}
m^{\Gamma}|_a=\sum_{k=1}^{l}\sum_{i>j}^{n}\mu_{j\rightarrow i}\Delta_{j\rightarrow i}(y_k^i-y_k^j),
\end{equation}
where $ \mu_{j\rightarrow i}$ is the mutation rate from nucleotide $j$ to nucleotide $i$ and $\Delta_{j\rightarrow i}$ is the energy difference of nucleotide $j$ to $i$, $\Delta_{j\rightarrow i}=\epsilon_j-\epsilon_i$. In the absence of any mutational bias, the mutation rates are equal, $\mu_{j\rightarrow i}=\mu/n-1$, where $\mu$ is the total mutation rate per position. Note that self mutations are excluded by definition. Inserting into the expression,
\begin{equation}
	m^\Gamma|_a=\frac{\mu}{n-1}\sum_{k=1}^{l}\sum_{i>j}^{n}(\epsilon_k^j-\epsilon_k^i)(y_k^i-y_k^j),
\end{equation}
we can multiply out the brackets to
\begin{equation}
		m^\Gamma|_a=\frac{\mu}{n-1}\sum_{k=1}^{l}\left((1-n)\sum_{j=1}^{n}\epsilon_k^jy_k^j+\sum_{i=1}^{n}\epsilon_k^i\sum_{j\neq i}^{n}y_k^j\right).
\end{equation}
Now we use $\sum_{i}y_k^i=1$, such that we can write $\sum_{j\neq i}^{n}y_k^j = 1 - y_k^i$,
\begin{align}
m^{\Gamma}|_a&=\frac{\mu}{n-1}\sum_{k=1}^{l}\left((1-n)\sum_{i=1}^{n}\epsilon_k^jy_k^j+\sum_{i=1}^{n}\epsilon_k^i(1-y_k^i)\right),\nonumber\\
&=\frac{\mu}{n-1}\sum_{k=1}^{l}\left(-n\sum_{i=1}^{n}\epsilon_k^jy_k^j+\sum_{i=1}^{n}\epsilon_k^i\right).
\end{align}
The mean trait value is given by $\Gamma=\sum_{k=1}^l\sum_{i=1}^{n}E_iy_k^j$, and the neutral mean value by $\Gamma_0=\frac{1}{n}\sum_{k=1}^l\sum_{i=1}^nE_i$, therefore the mutation drift from trailer mutations is given by
\begin{equation}
	m^{\Gamma}|_a=-\mu\frac{n}{n-1}\left(\Gamma-\Gamma_0\right)
\end{equation}
In our model driver mutations occur as substitutions, therefore the nucleotide frequencies in the trailer do not change. In general, with the new driver element, each possible nucleotide at that position can have a different contribution to the binding energy. This difference is weighted with the allele frequency of each nucleotide. The energy of nucleotide $i$ with driver element $\alpha$ is denoted by $E^\alpha_i$, the energy at position before the mutation by $E^*_i$. The number of possible driver elements is given by $\beta$, and the mutation rate of driver element $*$ to $\alpha$ by $\rho_{*\rightarrow\alpha}$. Then the contribution to the mutation drift from driver mutations is given by
\begin{equation}
m^\Gamma|_\mathbf{b}=\sum_{k=1}^{l}\sum_{\alpha=1}^{\beta}\sum_{i=1}^{n}\rho_{*\rightarrow\alpha}(E^\alpha_i-E^*_i)y_k^i.
\end{equation}
Again, we assume that there is no mutational bias, $\rho_{*\rightarrow\alpha}=\rho/(\beta-1)$. We can transform the sum to
\begin{align}
	&=-\frac{\rho}{\beta-1}(\beta\Gamma-\sum_{k=1}^{l}\sum_{\alpha=1}^{\beta}\sum_{j=1}^{n}\epsilon_\alpha^jy_k^i),\nonumber\\
	&=-\rho\frac{\beta}{\beta-1}(\Gamma-\Gamma_0'),
\end{align}
where $\Gamma_0'$ is the average over all possible binding energies with given trailer sequence. In general, this average is close to the neutral mean binding energy $\Gamma_0$. Then, the total mutation drift sums to
\begin{equation}
	m^\Gamma=-\left(\mu\frac{n}{n-1}+\rho\frac{\beta}{\beta-1}\right)\left(\Gamma-\Gamma_0\right).
\end{equation}
In the case if equal alphabet sizes of trailer and driver, the mutation drift takes the simple form
\begin{equation}
    m^\Gamma=-\frac{n}{n-1}\mu\left(1+\kappa\right)\left(\Gamma-\Gamma_0\right)\label{equ:mut_drift_end},
\end{equation}
with the non-equilibrium parameter $\kappa=\rho/\mu$. If there are different alphabet sizes, then rescaling one of the mutation rates will give the same result. This result can be confirmed by numeric simulations by observing the time evolution of the mean energy of a population over time in the absence of fitness. Then, the deterministic (neglecting stochastic fluctuations) time evolution of the mean is given by
\begin{equation}
    \dot\Gamma=m^\Gamma,
\end{equation}
which has the analytical solution
\begin{equation}
    \Gamma(t) = \Gamma_0 + (\Gamma_i - \Gamma_0)e^{-\frac{n}{n-1}\mu\left(1+\kappa\right)\, t}. 
\end{equation}
In figure \ref{fig:drift_scaling_n_rho} we can see (\purple{add figure description}).
\begin{figure}
    \centering
    \includegraphics[width=\textwidth]{drift_scaling_n_rho.pdf}
    \caption{Time evolution of mean and variance of populations in numeric simulations. Populations were initiated at $\Gamma_i=0$, and run for 50000 generations with $N=1000$, $\mu = 1/N$, $l=10$ and $\kappa= 0,\,0.5,\,1,\,5$ (blue, orange, green, red). Dashed lines show expected theoretical prediction.}
    \label{fig:drift_scaling_n_rho}
\end{figure}\\

\noindent In addition to the drift of the mean, we look at the scaling of the diversity within a population for various alphabet sizes and non-equilibrium ratios. From the simulations in figure \ref{fig:drift_scaling_n_rho}, we already see that the diversity in steady state is the same for all tested non-equilibrium ratios. Since a substitution in the driver sequence changes the preferred nucleotide for each species, there is no significant change in the diversity within the population. To study the dependence on the alphabet size, we have to consider the effect mutation rate, that is the rate at which the binding energy changes, which differs if the alphabet is bigger than binary. Given that the total mutation rate per position is $\mu$, the trait changing mutation rate is also $\mu$ if the position is a match, since any mutation will lead to a mismatch. However, if the position is a mismatch, then only one out of $n-1$ possible mutations leads to a trait change. Therefore, the average trait changing mutation rate $\tilde \mu$ is given by
\begin{equation}
    \tilde \mu = \frac{\Gamma}{\epsilon l} \frac{\mu}{n-1} + \left(1 - \frac{\Gamma}{\epsilon l} \right) \mu.
\end{equation}
In the absence of fitness, the steady state mean binding energy is $\Gamma/\epsilon=\Gamma_0\epsilon=1-1/n$, therefore the trait changing mutation rate is given by $\tilde \mu = 2\mu/n$. Then, the trait diversity within a population at steady state is given by 
\begin{equation}
    \Delta_E = \frac{2}{n}\mu Nl\epsilon^2
\end{equation}
In figure~\ref{fig:variance_scaling_n} the diversity within populations is shown as a function of alphabet size. The endpoints from the same simulations shown in figure~\ref{fig:drift_scaling_n_rho} fit the prediction well.

\begin{figure}
    \centering
    \includegraphics[width=75mm]{variance_scaling_n.pdf}
    \caption{Scaling of diversity within a population for various alphabet sizes and non-equilibrium ratios $\rho= 0,\,0.5,\,1,\,5$ (blue, orange, green, red). Dashed gray line shows theoretical prediction.}
    \label{fig:variance_scaling_n}
\end{figure}







\clearpage
\paragraph*{Supplementary Methods~3}
\label{supp3}
\textbf{Mutation-Selection-Balance and Non-Equilibrium.}\\ \\
The time evolution of the mean binding energy is given by \cite{held_survival_2019}

\begin{equation}
	\dot{\Gamma}=m^\Gamma+\Delta_E f'(\Gamma)+\chi_\Gamma(t),
\end{equation}

\noindent with $m^\Gamma = -\frac{n}{n-1}\mu(1+\kappa) (\Gamma - \Gamma_0)$ and $\Delta_E = \frac{2}{n}\mu Nl\epsilon^2$ (derived supplementary methods 2), where the mutation rate $\mu$ is the absolute mutation per position. The mutation selection balance (MSB) is the deterministic fix point of this equation $\dot{\Gamma}_{\mathrm{MSB}}=0$. With the mutation drift from equation \ref{equ:mut_drift_end}, we get
\begin{equation}
	0=-\frac{n}{n-1}\mu\left(1+\kappa\right)\left(\Gamma_{\mathrm{MSB}}-\Gamma_0\right)+\frac{2}{n}\mu N l \epsilon^2 f'(\Gamma_{\mathrm{MSB}}).
	\label{equ:zero}
\end{equation}
Note that the MSB is invariant under the limit of a vanishing mutation rate $\mu\rightarrow0$. At close to the upper plateau of the sigmoid fitness 
landscape the landscape can be approximated by an exponential landscape with derivative
\begin{equation}
\partial_\Gamma f(\Gamma, l) = -f_0\beta\exp\left[ \beta (\Gamma- (\Gamma_0-\Delta E ))\right],
\end{equation}
where $\Gamma_0$ is the mean energy in a population in the absence of selection. In the model of matches and mismatches, this is simply 
$\Gamma_0=\epsilon 3l/4$ in a four letter alphabet, where $\epsilon$ is the energy contribution of a mismatch.\\
Then, we can rewrite equation \ref{equ:zero} to retrieve the transcendental equation
\begin{equation}
	\Gamma_\mathrm{MSB}=\Gamma_0+\frac{2(n-1)}{n^2}\frac{1}{1+\kappa}Nl\epsilon^2\beta\exp\left[-\beta (\Gamma_\mathrm{MSB} -(\Gamma_0-\Delta E) )\right],
\end{equation}
which is solved by the ProductLog,
\begin{equation}
	\Gamma_\mathrm{MSB}(l)=\Gamma_0 - \frac{1}{\beta}\productlog\left(2 \frac{n-1}{n^2}\frac{ N f_0 l \epsilon^2 \beta^2e^{\beta\Delta E}}{1+\kappa}\right)\label{equ:msb_solution}.
\end{equation}
We can rewrite this result in terms of relative number of mismatches $\gamma/\epsilon l$,
\begin{equation}
\gamma_\mathrm{MSB}(l) - \gamma_0= -\frac{1}{l\,\epsilon\,\beta} \productlog\left(2 \frac{n-1}{n^2}\frac{ N f_0 l \epsilon^2 \beta^2e^{\beta\Delta E}}{1+\kappa}\right). \label{equ:intensive_scaling}
\end{equation}
Although the ProductLog scales with the binding site length $l$ and the non-equilibrium ratio $\kappa$, we can treat it as a constant,
\begin{equation}
	l_0 = \frac{1}{\beta\epsilon}\productlog\left(2 \frac{n-1}{n^2}\frac{ N f_0 l \epsilon^2 \beta^2e^{\beta\Delta E}}{1+\kappa}\right).\label{equ:l0_def}
\end{equation}
Now we can write
\begin{equation}
\gamma_\mathrm{MSB}(l) - \gamma_0\approx -\frac{l_0}{l}. \label{equ:intensive_scaling2}
\end{equation}
Having equation~(\ref{equ:msb_solution}) in hand, we can solve equation~(\ref{equ:zero}) for the fitness derivative at the Mutation-Selection-Balance,
 $f'\left(\Gamma_\mathrm{MSB}(l)\right)=\hat{f}(l)$,
 \begin{equation}
	\hat{f}(l) = -\frac{1}{\epsilon}\frac{n^2}{2(n-1)}\frac{1+\kappa}{N l}l_0.
\end{equation}
 In the exponential fitness landscape, the scaled genetic load is given by the derivative of the fitness landscape in respect ot the energy and the length fitness cost,
\begin{equation}
	\mathcal{L}= 2N\left(-\frac{f'(\Gamma)}{\beta}+f_l\,l.\right).
\end{equation}
Also we know, that the deterministic load is equal to the average load,
\begin{equation}
	\mathcal{L}= 2N\left(-\frac{\hat{f}(l)}{\beta}+f_l\,l\right)
\end{equation}
Finally, we can write the genetic load as a function of the driving parameter $\kappa$ and the binding length,
\begin{equation}
	\mathcal{L}(l,\kappa)=\frac{n^2}{n-1}\frac{1+\kappa}{\beta\epsilon}\frac{l_0}{l}+\lambda\,\frac{l}{l_0},
\end{equation} 
with the scaled fitness cost $\lambda=2Nf_l/l_0$. This load is minimized for given parameters at length
\begin{equation}
	l_\mathrm{opt} = l_0\sqrt{\frac{n^2}{\epsilon\beta \lambda (n-1)}(1+\kappa)}.
\end{equation}
The parameter $\lambda$ can be found by fixing the optimal length in equilibrium $l_\mathrm{opt}(\kappa=0)=10$.




\clearpage
\paragraph*{Supplementary Methods~4} \label{supp4}
{\bf{Dynamical Length Evolution.}}\\ \\

\noindent The slow dynamics of binding site length evolution leads to an asymmetry. As shown in equation~(\ref{equ:intensive_scaling2}), the conditional stationary distribution $Q(k|l)$ is peaked at $\gamma-\gamma_0=l_0/l$. We compute the selection coefficients of length mutations in first order, using the expansion of the exponential fitness landscape
\begin{equation}
	F(\Gamma, l) = F(\Gamma', l') + (\Gamma-\Gamma') f'(\Gamma', l') + (l-l') \left(\frac{3}{4}\epsilon f'(\Gamma', l') - f_l\right),
\end{equation}
where we used $f'(\Gamma', l') = \partial_\Gamma F(\Gamma', l') = 4/3\epsilon \,\partial_l F(\Gamma', l')$.
The selection coefficients for length mutations are the given by,
\begin{alignat}{3}
	& s_+^M    \quad && = F(\Gamma, l+1) - F(\Gamma, l)          \quad && = -f_l - \frac{3}{4}\epsilon\,f'(\Gamma, l),\\
	& s_+^{MM} \quad && = F(\Gamma+\epsilon, l+1) - F(\Gamma, l) \quad && = -f_l + \frac{1}{4}\epsilon\,f'(\Gamma, l),\\
	& s_-^M    \quad && = F(\Gamma, l) - F(\Gamma, l+1)          \quad && = f_l + \frac{1}{4}\epsilon\,f'(\Gamma, l),\\
	& s_-^{MM} \quad && = F(\Gamma-\epsilon, l) - F(\Gamma, l+1) \quad && = f_l - \frac{3}{4}\epsilon\,f'(\Gamma, l),
\end{alignat}
where the subscript is indicating increase (+) or decrease mutation (-) and the superscript is indicating whether the mutation is regarding a match($M$) or mismatch($MM$)
When adding a position, a match is added with probability 1/4, and mismatch with probability 3/4. The average selection coefficient of a length increase mutation is then given by

\begin{equation}
	\bar s_+ = \frac{1}{4} s_+^M + \frac{3}{4} s_+^{MM} = -f_l.
\end{equation}

\noindent When removing a position, the probability of removing a match is equal to the ratio of matches in the binding site $1-\gamma$, hence the average selection coefficient for a decrease mutation is given by 

\begin{equation}
\bar s_- = \gamma s_-^{MM} + (1-\gamma) s_-^{M} = f_l - (\gamma - \gamma_0)f'(\Gamma, l)\epsilon.
\end{equation}

\noindent Now we can insert equations~(\ref{equ:intensive_scaling2}) and (\ref{equ:fitness_der_scaling}). Then, the scaled selection coefficients $\sigma = 2Ns$ are given by
\begin{align}
	\sigma_+ &= -\frac{\lambda}{l_0},\\
	\sigma_- &= \frac{\lambda}{l_0} - 2\left(\frac{l_0}{l}\right)^2\frac{n-1}{n}(1+\kappa)\beta^2\epsilon^2.
\end{align}

\noindent Here we can see the asymmetry of length mutations. Length increase mutations are only constrained by the small fitness cost of each position, with $\sigma_-\sim 1/l_0$ (near-neutral evolution), while length decrease mutations  are marginally constrained by conservation of site function with $\sigma_+ \sim l_0^2/l^2\sim 1$ (marginal selection). We can write the selection coefficients as derivatives of the load,

\begin{align}
	\sigma_+ &= -\frac{\partial \mathcal{L}_l}{\partial l},\\
	\sigma_- &= l_0\frac{\partial \mathcal{L}_\kappa}{\partial l}\purple{\epsilon\beta} +\frac{\partial \mathcal{L}_l}{\partial l}.
\end{align}

\noindent \purple{We have this extra factor of $\epsilon\beta$, which is a factor of 2. Let's see if we can get rid of it somehow? Equation \ref{equ:intensive_scaling2} might be the right equation to do so.} Due to the separation of timescales, we can compute a steady state distribution of binding lengths $P(l)$. This distribution is obeying detailed balance, therefore we can compute

\begin{equation}
	\frac{P(l)}{P(l+1)}= \frac{u_-(l+1)}{u_+(l)},
\end{equation}

\noindent where $u_+$ is the length increase substitution rate and $u_-$ is the length decrease mutation rate. In first order we can rate the ratio of the to rates as 

\begin{align}
	\frac{u_-(l+1)}{u_+(l)} &= \frac{\exp(\sigma_+/2)}{\exp(\sigma_-/2)} + \mathcal{O}(\sigma^2),\\
	&\simeq\exp\left[ -\frac{l_0\beta\epsilon}{2}\left( \mathcal{L}_\kappa(l+1) - \mathcal{L}_\kappa(l) \right) - \left( \mathcal{L}_\lambda(l+1) - \mathcal{L}_\lambda(l) \right)\right],\\
	&=\exp\left[ \mathcal{F}_\mathrm{eff}(l+1) - \mathcal{F}_\mathrm{eff}(l) \right],
\end{align}

\noindent which the scaled effective potential

\begin{equation}
	\mathcal{F}_\mathrm{eff}(l) = -\epsilon\beta\frac{l_0}{2}\mathcal{L}_\kappa(l) - \mathcal{L}_\lambda (l).
\end{equation}

\noindent These rates define an equilibrium distribution 

\begin{equation}
	Q(l) \sim \exp\left[ \mathcal{F}_\mathrm{eff}(l) \right],
\end{equation}

\noindent which is peaked at 

\begin{equation}
	l_{opt} = l_0\epsilon\beta\sqrt{\frac{l_0}{2}\frac{2(n-1)}{n\lambda}(1+\kappa)}.
\end{equation}


%\begin{figure}
%	\centering
%	\includegraphics[]{supp5_l_opt_comparison.pdf}
%	\caption{}
%	\label{fig:supp5_opt_length_comp}
%\end{figure}




\clearpage
\outcomment{
To compute length substitution rates, we have to consider all possible energies at which a length mutation can occur, since at each energy there is different 
selection coefficient. Added elements are random and therefore the probability of adding a match is $1/4$. The 
probability of a match being removed depends on the ratio of matches in the binding site, $k/l$.
\begin{align}
u_+(l)=&N\nu\sum_{k}\left[\frac{1}{4}p\left(s^+_m(k,l)\right)+\frac{3}{4}p\left(s^+_{mm}(k,l)\right)\right]Q(k|l),\label{equ:up_rate}\\
u_-(l)=&N\nu\sum_{k}\left[\frac{k}{l}p\left(s^-_m(k,l)\right)+\left(1-\frac{k}{l}\right)p\left(s^-_{mm}(k,l)\right)\right]Q(k|l),\label{equ:down_rate}
\end{align}
where $\nu$ is the length mutation rate, $N$ is the population and $p(s)$ the Kimura fixation probability for selection coefficient $s$. The selection coefficients are
\begin{align*}
	s_m^+&=f(k+1,l+1)-f(k,l) &\text{match addition},\\
	s_{mm}^+&=f(k,l+1)-f(k,l) &\text{mismatch addition},\\
	s_m^-&=f(k-1,l-1)-f(k,l) &\text{match removal},\\
	s_{mm}^-&=f(k,l-1)-f(k,l) &\text{mismatch removal}.
\end{align*}
Between each length mutation, the binding sites relax to the steady state distribution for fixed length, $Q(k|l)$ (\textcolor{blue}{hier vielleicht plot von steady state distribution und k's vor mutations}). With the substitution rates in hand, we can compute a steady state distribution $P(l)$ for binding lengths using detailed balance (\textcolor{blue}{steady state = equilibrium in 1D}),
\begin{equation}
u_+(l)P(l)=u_-(l+1)P(l).\label{equ:det_bal}
\end{equation}
The stationary length distribution is also the result of a marginalization of the two dimensional probability,
\begin{align}
&Q(k,l)=\frac{1}{Z}Q_0(k,l)\exp\left[2NF(k,l)\right]\\
&\Rightarrow P(l) = \frac{1}{Z}\sum_{k=0}^{l}Q(k,l).
\end{align}
Evolutionary distributions can usually be written as exponential of a potential, $P(l)\sim e^{-\beta F_\mathrm{eff}(l)}$, which we can compute using the substitution rates,
\begin{equation}
	F_\mathrm{eff}(l)=\sum_{i=1}^{l-1}\log\frac{u_+(i)}{u_-(i+1)}+c.
\end{equation}
In the absence of any fitness landscape, $f_l=0$ and $f_0=0$, there is no preferred binding length, and the length distribution is uniform, $P(l)=P_0(l)=1/Z$. In the case of neutral binding, i.e., $f_0=0$, the fitness landscape is simply linear in length. Therefore the length distribution is simply an exponential distribution, following from (\textcolor{blue}{find citation})
\begin{equation}
	P(l)=P_0(l)\exp(2NF(l))\sim \exp(-2Nf_l\,l)
\end{equation}

\vspace{3cm}
\textcolor{purple}{Now we do the whole thing again, but using the approximation using only the predicted peak from the previous section. This allows us to neglect the part of the distribution which is on the lower fitness plateau and anyways not part of the previous calculation,
\begin{align}
u_+(l)=&N\nu \left[ \frac{1}{4} p\left(s^+_m \right) + \frac{3}{4} p\left(s^+_{mm} \right) \right], \label{equ:up_rate2}\\
u_-(l)=&N\nu\left[(1-\gamma) p\left(s^-_m \right) + \gamma p\left(s^-_{mm}\right)\right].\label{equ:down_rate2}
\end{align}
We computed the selection coefficients up to second order in the previous section, and are going to use them here to compute the forward and backward rates.
}
}

\end{document}
