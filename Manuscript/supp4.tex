\textbf{Dynamical Length Evolution.}\\

\noindent The slow dynamics of binding site length evolution leads to an asymmetry. As shown in equation~(\ref{equ:intensive_scaling}), the conditional stationary distribution $Q(k|l)$ is peaked at $\gamma \approx \gamma_0 - l_0/l$. We compute the selection coefficients of length mutations in first order, using the expansion of the exponential fitness landscape
\begin{equation}
	F(\Gamma, l) = F(\Gamma', l') + (\Gamma-\Gamma') f'(\Gamma', l') + (l-l') \left(\frac{n-1}{n}\epsilon f'(\Gamma', l') - f_l\right),
\end{equation}
where we used $f'(\Gamma', l') = \partial_\Gamma F(\Gamma', l') = n/(n-1)\, \epsilon \,\partial_l F(\Gamma', l')$.
The selection coefficients for length mutations are the given by,
\begin{alignat}{3}
	& s_+^M    \quad && = F(\Gamma, l+1) - F(\Gamma, l)          \quad && = -f_l - \frac{n-1}{n}\epsilon\,f'(\Gamma, l),\\
	& s_+^{MM} \quad && = F(\Gamma+\epsilon, l+1) - F(\Gamma, l) \quad && = -f_l + \frac{1}{n}\epsilon\,f'(\Gamma, l),\\
	& s_-^M    \quad && = F(\Gamma, l) - F(\Gamma, l+1)          \quad && = f_l + \frac{1}{n}\epsilon\,f'(\Gamma, l),\\
	& s_-^{MM} \quad && = F(\Gamma-\epsilon, l) - F(\Gamma, l+1) \quad && = f_l - \frac{n-1}{n}\epsilon\,f'(\Gamma, l),
\end{alignat}
where the subscript is indicating increase (+) or decrease mutation (-) and the superscript is indicating whether the mutation is regarding a match($M$) or mismatch($MM$)
When adding a position, a match is added with probability $1/n$, and mismatch with probability $(n-1)/n$. The average selection coefficient of a length increase mutation is then given by
\begin{equation}
	\bar s_+ = \frac{1}{n} s_+^M + \frac{n-1}{n} s_+^{MM} = -f_l.
\end{equation}
When removing a position, the probability of removing a match is equal to the ratio of matches in the binding site $1-\gamma$, hence the average selection coefficient for a decrease mutation is given by 
\begin{equation}
\bar s_- = \gamma s_-^{MM} + (1-\gamma) s_-^{M} = f_l - (\gamma - \gamma_0)f'(\Gamma, l)\epsilon.
\end{equation}
Now we can insert equations~(\ref{equ:intensive_scaling}) and (\ref{equ:fitness_deriv}). Then, the scaled selection coefficients $\sigma = 2Ns$ are given by
\begin{align}
	\sigma_+ &= -\frac{\lambda}{l_0},\\
	\sigma_- &= \frac{\lambda}{l_0} - \left(\frac{l_0}{l}\right)^2\frac{n^2}{n-1}(1+\kappa).
\end{align}
Here we can see the asymmetry of length mutations. Length increase mutations are only constrained by the small fitness cost of each position, with $\sigma_-\sim 1/l_0$ (near-neutral evolution), while length decrease mutations  are marginally constrained by conservation of site function with $\sigma_+ \sim l_0^2/l^2\sim 1$ (marginal selection). We can write the selection coefficients as derivatives of the load,
\begin{align}
	\sigma_+ &= -\frac{\partial \mathcal{L}_l}{\partial l},\\
	\sigma_- &= -\epsilon\beta l_0\frac{\partial \mathcal{L}_\kappa}{\partial l} +\frac{\partial \mathcal{L}_l}{\partial l}.
\end{align}
Due to the separation of timescales, we can compute a steady state distribution of binding lengths $P(l)$. This distribution is obeying detailed balance, therefore we can compute
\begin{equation}
	\frac{P(l)}{P(l+1)}= \frac{u_-(l+1)}{u_+(l)},
\end{equation}
where $u_+$ is the length increase substitution rate and $u_-$ is the length decrease mutation rate. In first order we can rate the ratio of the to rates as 
\begin{align}
	\frac{u_-(l+1)}{u_+(l)} &= \frac{\exp(\sigma_+/2)}{\exp(\sigma_-/2)} + \mathcal{O}(\sigma^2),\\
	&\simeq\exp\left[ -\frac{\epsilon\beta l_0}{2}\left( \mathcal{L}_\kappa(l+1) - \mathcal{L}_\kappa(l) \right) - \left( \mathcal{L}_\lambda(l+1) - \mathcal{L}_\lambda(l) \right)\right],\\
	&=\exp\left[ \mathcal{F}_\mathrm{eff}(l+1) - \mathcal{F}_\mathrm{eff}(l) \right],
\end{align}
which the scaled effective potential
\begin{equation}
	\mathcal{F}_\mathrm{eff}(l) = -\frac{\epsilon\beta l_0}{2}\mathcal{L}_\kappa(l) - \mathcal{L}_\lambda (l).
\end{equation}
These rates define an equilibrium distribution 
\begin{equation}
	Q(l) \sim \exp\left[ \mathcal{F}_\mathrm{eff}(l) \right],
\end{equation}
which is peaked at 
\begin{equation}
	l_{opt} = l_0^{3/2}\sqrt{\frac{n^2}{ 2\lambda (n-1)}(1+\kappa)}.
\end{equation}

\begin{figure}
	\centering
	\includegraphics[width=0.9\textwidth]{optim_l_dynamic_with_sim.pdf}
	\caption{Optimal length for varying maximal fitness values. Blue is predicted curve, orange dots are simulations from substitution dynamics. Error bars are standard deviation from the mean.}
	\label{}
\end{figure}

